% !TeX root = ../root.tex
% -*- root: ../root.tex -*-
\subsection{Safety Goals}

Safety Goals: The chapter defines safety goals for the behavior arbitration method, addressing fault tolerance and traffic safety for automated driving applications.

\begin{itemize}
    \item Safety Goal 1: System Covers Necessary Behavior Competencies: The behavior generation must cover sufficient behavior competencies to handle expected situations within the ODDs. If an unsupported situation arises, the system must recognize it and execute an MRM to transition the vehicle to an MRC.
    \item Safety Goal 2: System Ensures MRM When Leaving ODDs: If the ODDs are carefully specified and comprehensively covered by behavior competencies, they may still be violated due to weather changes or unexpected changes in the scenery. In such cases, the system must recognize this and transition the vehicle to a safe state as described in Section 4.2.
    \item Safety Goal 3: System is Robust Against Behavior Module Failures: Arbitrators follow a comparatively simple logic, and their robust and reliable execution can be ensured through component tests or even formal proofs. Behavior modules, on the other hand, cover complex tasks and may be implemented by different developers with varying quality. Therefore, rare failures of behavior modules in the form of software crashes, especially in the early development stage, cannot be ruled out and must be handled.
    \item Safety Goal 4: System Avoids Insufficient Planning Frequency: To ensure that the trajectories planned by the behavior modules can actually be realized, the entire behavior generation of a production vehicle must meet hard real-time conditions. The desired trajectory of the planning interval must be passed on to the controller no later than the time of the next planning interval.
    \item Safety Goal 5: System Prevents Invalid Actuator Outputs: The use of nonlinear optimization or machine learning methods within behavior modules may lead to trajectories that are kinematically or dynamically infeasible. Therefore, the behavior arbitration must be able to recognize and intercept trajectories that would lead to invalid actuator outputs.
    \item Safety Goal 6: System Avoids Risky Actuator Outputs: It must also be ensured that risky trajectories, i.e., those that could lead to critical driving situations, are avoided if possible. However, it is sufficient to show that the planned desired trajectory still allows a switch to the fail-safe trajectory in the next planning interval and that the latter is provably collision-free.
\end{itemize}

\subsection{Safety Concept}

Safety Concept: The safety concept builds on the concepts of fault tolerance and traffic safety presented in Sections 3.2 and 3.3, as well as on measures for real-time capability.
Safety Goals: The safety concept addresses the safety goals 1-6 by:

\begin{itemize}
    \item Safety Goal 1: System Design and Verification for Traffic Safety
    \item Safety Goal 2: Real-Time Capability
    \item Safety Goal 3: Redundancy and Diversity of Behavior Modules
    \item Safety Goal 4: Modular Design and Decoupling of Behavior Modules
    \item Safety Goal 5: Verification of Actuator Outputs
    \item Safety Goal 6: Fail-Safe Behavior Modules
\end{itemize}

\subsubsection*{Error Prevention in System Design}

\begin{itemize}
    \item Structured System Design: A structured system design contributes to error prevention with regard to traffic safety and thus forms the basis for achieving the stated safety goals in behavior generation.
    \item ODDs: The ODDs must first be defined for the overall system. This means clarifying where and under what conditions the automated vehicle is to be operated.
    \item Behavior Competencies: In an iterative process, the necessary behavior competencies are derived from the ODDs.
    \item Behavior Modules: Behavior modules are then designed, each covering one or more behavior competencies.
    \item ODD Overlapping: If the operating conditions of the behavior modules do not overlap significantly, it can only be switched between behavior modules in rare cases and thus not the entire ODDs are served.
    \item Detailed ODDs: Therefore, it makes sense to define detailed ODDs per behavior module in order to be able to examine and ensure their overlaps.
    \item Test Coverage: The ODDs of the behavior modules are also another starting point for creating a scenario catalog for validating the behavior modules.
    \item Residual Risk: A complete coverage of the ODDs and the necessary behavior competencies will not be possible due to the abundance and complexity, especially of very rare events.
    \item Fallback Level: The residual risk resulting from unforeseen situations is intercepted by a fallback level with verification of the fail-safe trajectories.
    \item Mandatory MRM: To ensure that an MRM is always available, the root of the arbitration graph must provide at least one MRM that is always applicable (e.g., an emergency stop).
\end{itemize}

\subsubsection*{Error Prevention Due to Real-Time Capability}

\begin{itemize}
    \item Real-Time Capability: To avoid insufficient planning frequency, real-time capability must be ensured.
    \item Latency: The latency of behavior arbitration, including the computation time of behavior planning, must be kept low.
    \item Parallel Processing: Parallel processing of behavior modules can be used to reduce latency.
    \item Limited Parallelization: Full parallelization is not scalable, so a limited number of alternative options are evaluated in parallel.
    \item Event- or Time-Trigger: Behavior decision should be triggered by a combination of event- and time-triggers.
    \item Soft Real-Time: In research vehicles, soft real-time requirements are sufficient.
    \item Planning Frequency: The planning frequency should be between 5 and 10 Hz.
    \item Processing Time: The processing time for behavior generation and decision should not exceed 200 ms.
    \item Timeout: Behavior modules are called with a timeout to diagnose insufficient reaction time.
    \item Monitored Command: The return value of the getCommand function is extended to a monitored command, including an error value.
    \item Redundancy and Diversity: Alternative options and the fail-safe trajectory can be used to compensate for errors.
\end{itemize}

Additional Notes:

\begin{itemize}
    \item The measures for error prevention due to real-time capability are intended to ensure that the behavior arbitration method can generate behavior plans in a timely manner.
    \item The measures are based on the principles of parallel processing, event-driven programming, and timeouts.
    \item The measures are implemented in a way that is compatible with the soft real-time requirements of research vehicles.
\end{itemize}

\subsubsection*{Event- or Time Trigger}

\begin{itemize}
    \item Event Trigger: The behavior decision is triggered by an input signal or event, such as the arrival of new sensor data.
    \item Time Trigger: A timer is used to ensure a minimum planning frequency.
    \item Combination of Triggers: A combination of event- and time-triggers is used to achieve the best of both worlds.
    \item Event-Based Triggering: Event-based triggering ensures low latency.
    \item Time-Based Triggering: Time-based triggering ensures a minimum planning frequency.
    \item Robustness: The use of a timer makes the system robust to disturbances in the input frequency.
\end{itemize}

\subsubsection*{Real-Time Capability}


\begin{itemize}
\item Hard Real-Time vs. Soft Real-Time: Hard real-time is required in industrial applications, while soft real-time is sufficient in research.
\item Real-Time Requirements: The real-time requirements for behavior arbitration are defined.
\item Timeout Mechanism: A timeout mechanism is used to detect and handle insufficient reaction time.
\item Error Compensation: Alternative options and the fail-safe trajectory can be used to compensate for errors.
\end{itemize}

Additional Notes:

\begin{itemize}
    \item The real-time capability measures are designed to meet the soft real-time requirements of research vehicles.
    \item The measures are based on a combination of parallel processing, event-driven programming, and timeouts.
    \item The measures are implemented in a way that is flexible and can be adapted to different applications.
\end{itemize}

\subsubsection*{Fail-safe due to error compensation}

    Failure Prevention Goals:
\begin{itemize}
    \item The behavior arbitration method should be robust against failures of behavior modules and invalid setpoints.
    \item The system should be able to handle transient and permanent faults.
    \item The system should be able to compensate for systematic design faults in behavior planning methods.
\end{itemize}
    Measures for Error Prevention:
\begin{itemize}
    \item Redundancy: Multiple instances of a behavior module can be used to compensate for failures.
    \item Diversity: Different behavior modules that address the same behavior competence can be used to compensate for failures.
    \item Heartbeat Monitoring: Each behavior module sends a signal to its arbitrator at regular intervals to indicate that it is still functional.
    \item Restart: If a behavior module fails, it can be restarted.
\end{itemize}
Exclusion: If a behavior module fails too frequently, it can be excluded from the item of behavior options.
\begin{itemize}
    \item Verification: Setpoints are verified for validity and feasibility.
    \item Compensation: Invalid or infeasible setpoints are compensated for by using alternative behavior options.
\end{itemize}
    Benefits of the Approach:
\begin{itemize}
    \item The measures are based on the principles of parallel processing, event-driven programming, and timeouts.
    \item The measures are implemented in a way that is flexible and can be adapted to different applications.
    \item The measures are effective in preventing errors due to real-time capability and invalid setpoints.
\end{itemize}

Additional Notes:

\begin{itemize}
    \item The use of redundancy and diversity is particularly well-suited for behavior modules that use randomized procedures or initializations in trajectory planning.
    \item Diversity is effective in compensating for design faults in behavior planning methods.
    \item The heartbeat mechanism ensures that failures are detected quickly.
    \item Restarting a behavior module can be an effective way to recover from a transient fault.
    \item Excluding a behavior module that fails too frequently can prevent further errors.
    \item Verification of setpoints helps to prevent invalid or infeasible maneuvers.
    \item The combination of error prevention measures ensures that the behavior arbitration method is robust against a wide range of faults.
\end{itemize}

\subsubsection*{Road safety through verification}

\begin{itemize}
\item Safety Goal: The behavior arbitration method should generate safe maneuvers.
\item Verification Method: Reachability analysis is used to verify the safety of trajectories.
\item Error Diagnosis: Risky maneuvers are detected by comparing the ego-occupancy of the planned fail-safe trajectory with the object worst-case occupancies.
\item Error Handling: If a maneuver is not provably collision-free, the arbitrator selects an alternative option or falls back to a specialized safety maneuver.
\item Benefits of the Approach:

\begin{itemize}
    \item The method is able to detect and prevent risky maneuvers.
    \item The method is based on a well-established mathematical framework.
    \item The method is compatible with different behavior planning methods.
    \item The method can be extended to consider other safety aspects.
\end{itemize}

\end{itemize}

Additional Notes:

\begin{itemize}
    \item The reachability analysis method assumes that all objects will move according to their predicted trajectories.
    \item The method does not take into account the uncertainty of sensor measurements.
    \item The method can be computationally expensive, especially for long planning horizons.
    \item Despite these limitations, reachability analysis is a valuable tool for verifying the safety of autonomous vehicle behaviors.
\end{itemize}

\subsection{System Overview}

\begin{itemize}
\item Generic Verifying Arbitrator: The behavior arbitration algorithm is extended to include a verification step to ensure that only safe maneuvers are executed.
\item Verification Method: The verification method is based on reachability analysis and takes into account the worst-case behavior of other objects in the environment.
\item Error Handling: If a maneuver is not provably safe, the arbitrator selects an alternative option or falls back to a specialized safety maneuver.
\item Benefits of the Approach:

\begin{itemize}
    \item The method is able to detect and prevent risky maneuvers.
    \item The method is based on a well-established mathematical framework.
    \item The method is compatible with different behavior planning methods.
    \item The method can be extended to consider other safety aspects.
\end{itemize}

\end{itemize}
