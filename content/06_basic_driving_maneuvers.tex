% !TeX root = ../root.tex
% -*- root: ../root.tex -*-
\section{Basic Driving Maneuvers}
\subsection{Unparking}
Goal: Move the vehicle from a parking space to a suitable position on a lane.
Start Condition:

\begin{itemize}
    \item The vehicle is stationary.
    \item The vehicle is located in a parking lot or other open area that adjoins the starting lane.
    \item The vehicle is surrounded by a free, sufficiently large maneuvering area.
    \item The maneuvering area and a sufficiently long section of the starting lane are visible and free of other traffic participants.
\end{itemize}

Continuation Condition:

\begin{itemize}
    \item The vehicle is still in the parking lot, open area, or starting lane.
    \item The maneuvering area and a sufficiently long section of the starting lane are visible and free of other traffic participants.
    \item The target position has not been reached with sufficient accuracy.
\end{itemize}

Behavior Generation:

\begin{itemize}
    \item A three-stage approach can be used to generate simple and space-saving trajectories:
          \begin{itemize}
              \item Connect start and target pose using Reeds-Shepp rules.
              \item If not feasible, use a Reeds-Shepp path with two direction changes.
              \item If still not feasible, use hybrid A*-graph search.
          \end{itemize}
    \item Extend the A*-graph search with motion primitives to consider uncertainties in localization and control.
    \item Set the turn signal as part of the HMI.
\end{itemize}

\subsection{Parking}

Goal: Park the vehicle in a parking space or garage.
Start Condition:

\begin{itemize}
    \item The vehicle is nearly stationary.
    \item The vehicle is located on a target lane that adjoins a parking space or suitable open area.
    \item The vehicle is surrounded by a sufficiently large maneuvering area.
    \item The maneuvering area is visible and free of other traffic participants.
\end{itemize}

Continuation Condition:

\begin{itemize}
    \item The vehicle is still on the target lane or the selected parking space.
    \item The maneuvering area is visible and free of other traffic participants.
    \item The target parking position has not been reached with sufficient accuracy.
\end{itemize}

Behavior Generation:

\begin{itemize}
    \item Use methods similar to parking, such as deterministic Reeds-Shepp paths and hybrid A*-graph search.
    \item Determine the longitudinal behavior along the path using direct optimization of a multiple integrator under quadratic costs.
    \item Set the turn signal as part of the HMI.
\end{itemize}


\subsection{Following a Lane}

Goal: Follow the current lane and preceding traffic.
Start Condition:

\begin{itemize}
    \item The ego vehicle is located within a lane of the route.
    \item It is not relevant whether other vehicles are in the same lane or not.
\end{itemize}

Continuation Condition:

\begin{itemize}
    \item The ego vehicle is still located within a lane of the route.
\end{itemize}

Behavior Generation:

\begin{itemize}
    \item Determine the current lane corridor based on the current position and selected route.
    \item Use nonlinear optimization to determine the desired trajectory within the corridor.
    \item Consider preceding traffic and stopping lines.
    \item If a lane change is necessary but cannot be executed, end the lane corridor virtually at the last possible position for the lane change.
    \item Use the stopping maneuver at the end of the route to initiate parking.
\end{itemize}

Additional Notes:

\begin{itemize}
    \item The follow-up behavior is one of the most frequently investigated driving maneuvers.
    \item In the longitudinal behavior, the follow-up can be carried out both with and without preceding traffic.
    \item An interesting question is the end of a lane: Should the follow-up behavior also be able to be called or continued if the lane ends shortly?
    \item In this work, stopping maneuvers at the end of a lane are also implemented as part of the follow-up.
    \item The start and continuation conditions remain true in this case.
    \item The inclusion of stopping maneuvers in the follow-up behavior means that the vehicle is brought to a stop at the end of the route and the parking behavior can then be taken over.
    \item In places where a lane change is necessary to follow the route, but cannot currently be executed, the lane corridor of the follow-up ends virtually at the last possible position for the lane change.
\end{itemize}


\subsection{Crossing an Intersection}

    Goal: Pass an intersection safely and legally.
    Start Condition:

\begin{itemize}
    \item The ego vehicle is located within a lane of the route.
    \item There is a stop line with associated right-of-way lanes in the lane corridor in front of the ego vehicle.
    \item These lanes are sufficiently visible and the associated conflict zones are free of predictions of right-of-way traffic participants.
\end{itemize}

    Continuation Condition:

\begin{itemize}
    \item The ego vehicle is still located within a lane of the route.
    \item The right-of-way lanes associated with the current stop line are sufficiently visible, taking into account hysteresis.
    \item The current stop line and corresponding conflict zones have not yet been crossed.
    \item None of the predictions of right-of-way traffic participants suggest that such a traffic participant will reach a conflict zone before the ego vehicle has left it.
\end{itemize}

    Behavior Generation:

\begin{itemize}
    \item Determine the lane corridor as in the follow-up behavior, but do not cut off the corridor at the current stop line.
    \item Use nonlinear optimization to determine the desired trajectory within the corridor.
    \item Deactivate the virtual stop line in the longitudinal behavior to cross it.
\end{itemize}

Additional Notes:

\begin{itemize}
    \item This behavior module takes over the decision of whether and when the intersection can be passed.
    \item Since the follow-up behavior already stops at priority-related stop lines, the task of stopping at this point is actually left to the follow-up behavior module.
    \item The decision basis is defined explicitly in the start condition, which contributes to good comprehensibility.
    \item In contrast to the follow-up behavior, the cross-behavior deactivates the corresponding virtual stop line in the longitudinal behavior in order to cross it.
    \item If the ego vehicle is within a planning horizon of an intersection, the cross-behavior module checks whether the ego lane conflicts with other lanes and whether it can be expected that right-of-way traffic participants will pass through the associated conflict zone.
    \item In the case of several consecutive right-of-way scenarios, the cross-behavior module only considers the right-of-way situation of the next stop line (referred to as the current stop line in the following).
    \item In order to avoid oscillating behavior at objects on the decision boundary, the decision boundary of the continuation condition should lie closer to the intersection in the style of hysteresis than the start condition.
    \item The lane corridor is determined by the cross-behavior module as in the follow-up behavior, i.e. by concatenating successive lanelets, with the difference that the corridor is then not cut off at the current stop line.
    \item At subsequent stop and yield lines, the corridor is again cut off regularly.
\end{itemize}

\subsection{Slowly Pass Hazard}

    Goal: Approach and pass a dangerous spot at a reduced speed.
    Start Condition:

\begin{itemize}
    \item The ego vehicle is located within a lane of the route.
    \item A dangerous spot is located within the lane corridor.
    \item The lane is not completely blocked.
    \item In the case of a crosswalk, there are no VRUs in the vicinity.
\end{itemize}

    Continuation Condition:

\begin{itemize}
    \item The ego vehicle is still located within a lane of the route.
    \item A dangerous spot is located within the lane corridor.
    \item The lane is not completely blocked.
    \item In the case of a crosswalk, there are no VRUs in the vicinity.
\end{itemize}

    Behavior Generation:

\begin{itemize}
    \item Take over the lane corridor from the follow-up behavior module.
    \item Adjust the desired speed at the dangerous spot.
    \item Determine the trajectory using the same method as in the follow-up behavior, e.g. nonlinear optimization.
    \item If appropriate, activate the hazard lights as HMI.
\end{itemize}

Additional Notes:

\begin{itemize}
    \item This behavior module is a specialization of the follow-up behavior module.
    \item The start condition of the "Dangerous Spot Slowly Crossing" behavior module checks, in addition to the start condition of the follow-up behavior, whether a dangerous spot is present within the planning horizon of the current lane.
    \item The continuation condition of the "Dangerous Spot Slowly Crossing" behavior module is the same as its start condition.
    \item The lane corridor is taken over by the "Dangerous Spot Slowly Crossing" behavior module from the follow-up behavior module, but the desired speed stored at the dangerous spot is adjusted.
    \item It may also be appropriate to activate the hazard lights as HMI.
\end{itemize}

\subsection{Change Lanes}

    Goal: Change lanes to the target lane.
    Start Condition:

\begin{itemize}
    \item The ego vehicle is located within a lane of the route.
    \item A neighboring lane exists in the target direction.
    \item A lane change to this target lane is permitted.
    \item Any traffic participants on this lane are at a sufficient spatial and temporal distance from the ego vehicle.
\end{itemize}

    Continuation Condition:

\begin{itemize}
\item The ego vehicle is still located within the start and/or target lane.
        Any traffic participants on the target lane are at a sufficient spatial and temporal distance from the ego vehicle.
\item The ego vehicle has completely reached the target lane, or in the case of an abort, the start lane.
\end{itemize}

    Behavior Generation:

\begin{itemize}
    \item The lane corridor for the lane change consists of three consecutive segments in the naive approach:
          \begin{itemize}
              \item First, it follows the start lane exclusively.
              \item Then, the segment of the actual lane change includes both lanes.
              \item Finally, the corridor follows only the target lane.
          \end{itemize}
    \item Two additional segments are inserted at the beginning and end of the middle corridor segment to ensure a continuous transition of the corridor boundaries and thus also of the centerline.
    \item A trajectory is then planned within this lane corridor, e.g., using nonlinear optimization.
    \item The lengths of the individual segments are parameterizable in order to be able to optimally coordinate them with each other depending on the application and the trajectory planner used.
    \item Stop and yield lines are also taken into account during lane changes, e.g., to avoid approaching a red light on the target lane too quickly.
\end{itemize}

Additional Notes:

\begin{itemize}
    \item This behavior module is initially defined independently of the direction and is instantiated once as "Lane Change Left" and "Lane Change Right" for each application.
    \item If there are other road users on the target lane, it must be ensured that an appropriate distance is maintained to both the preceding and following object during the lane change.
    \item The purpose of a lane change is to bring the ego vehicle onto the target lane, so the behavior module should continue until the ego vehicle has completely arrived on the target lane.
    \item In a naive implementation, the corridor boundaries and consequently also the centerline would jump at the beginning and end of the middle segment.
    \item Since trajectory planners often use this centerline as a reference, this discontinuity would impair or even prevent the convergence of the planner.
    \item Stop and yield lines are also considered during lane changes to avoid, for example, approaching a red light on the target lane too quickly.
\end{itemize}
