\section{main}

\begin{itemize}
    \item Motivate safety layers by presenting a functional but not robust/safe arbitration graph capable of playing Pacman.
    \item Improve this implementation by adding safety layers.
    \item Split this into two sections: Unreliable and Unsafe
\end{itemize}

\subsection{Unreliable}
Issues that arise due to bad implementations etc. → Error handling
\begin{itemize}
    \item Basic arbitration graph that works in theory
    \item Might contain unsafe or unreliable behavior implementations
    \item Contains Chase, RunAway and Collect (EatClosest + GoToPatch) behaviors
    \item Example: EatClosest contains bug and throws exception when two or more closest pills exist
    \item Reference to paper title/research question: How to mitigate unsafe and unreliable behavior implementations?
    \item We could fix the bug in EatClosest but we want to show how to handle these scenarios in a more general way using Arbitration Graphs
    \item → Detect error and add RandomWalk as first fallback layer
\end{itemize}

\subsection{Unsafe}
Issues that arise due to logic errors. Output is executable but considered invalid or unsafe.
\begin{itemize}
    \item Random walk could output commands crashing into walls
    \item Domain specific verification step detects this → List examples from other domains
    \item Add dumb behavior (preferably so easy that it is proofably safe) as last resort fallback layer → StayInPlace
    \item StayInPlace does not have to pass verifier so Arbitration Graphs will always return a commnd
\end{itemize}
Discussion: StayInPlace could still be undesirable → Add additional fallback layers such as a duplicated RunAway (probably using other parameter).
