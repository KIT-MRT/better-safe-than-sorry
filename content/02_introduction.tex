% !TeX root = ../root.tex
% -*- root: ../root.tex -*-
\section{Introduction}

\subsection{Motivation}

Behavior planning and decision-making are critical components that enable robots to perform tasks autonomously and adaptively in dynamic environments.
These processes involve selecting and executing commands to achieve specific goals while responding to changes and uncertainties in the surroundings.
Ensuring safety and robustness in behavior planning and decision-making is essential for robots to operate reliably and effectively across various applications, such as autonomous driving, industrial automation, and service robotics.

Arbitration graphs, a type of hierarchical behavior model, can be used to manage complex decision-making processes in robots.
The bottom-up architecture makes them very scalable and maintainable while allowing for a very transparent decision-making process.
The modularity of arbitration graphs enables the integration of diverse scenario-specific methods and seamless switching between them.
However, the inherent complexity and dynamic nature of real-world environments pose significant challenges to the safety and robustness of these systems.

Ensuring the safety and robustness of the arbitration graphs governing such decisions is paramount to prevent failures and ensure reliable operation under varying conditions.

This paper focuses on enhancing the safety and robustness of arbitration graphs by identifying and handling erronous or unsafe commands at runtime.

\subsection{State of the art}
\todo[inline]{Add section about the need for decision-making and end-to-end learning vs rule-based systems.}

\begin{figure}
    \centering
    \includegraphics[width=0.5\textwidth]{figures/entt_pacman_with_graph.png}
    \caption{The Pac-Man implementation used to demonstrate the proposed method and the corresponding arbitration graph.}
    \label{fig:entt-pacman}
\end{figure}
\subsection{Pac-Man}

Pac-Man is a classic arcade game released in 1980, where the player controls Pac-Man, a yellow, circular character navigating a maze.
The primary objective is to eat all the dots in the maze while avoiding four ghosts—Blinky, Pinky, Inky, and Clyde—that pursue Pac-Man.
Consuming power pellets temporarily turns the ghosts blue, allowing Pac-Man to eat them for bonus points.
Depicted in Figure \ref{fig:entt-pacman} is an open-source implementation\footnote{github.com/indianakernick/EnTT-Pacman} of the game that we use to demonstrate the proposed method.
We split Pac-Man's behavior into the following behavioral components:

\begin{itemize}
    \item \textbf{Avoid Ghosts:} Pac-Man tries to increase the distance to the ghosts to avoid being eaten.
    \item \textbf{Chase Ghosts:} After consuming a power pellet Pac-Man might try to eat the ghosts for extra points.
    \item \textbf{Eat Closest Dot:} Pac-Man moves towards the dot closest to him.
    \item \textbf{Change Dot Cluster:} Pac-Man moves towards another area with a high dot density.
\end{itemize}

The arbitration graph in Figure \ref{fig:entt-pacman} visualizes the corresponding arbitration graph.

\subsection{Contributions}

\begin{itemize}
    \item \textbf{Verification logic:} The arbitration process is extended by a verification logic that ensures that only behavior options are executed that pass verification. This verification logic is based on three verifiers that check potential maneuvers for validity, feasibility, and traffic safety.
    \item \textbf{Safety concept:} A safety concept is designed to ensure robust and safe behavior arbitration through verification, real-time capability, redundancy, and diversity.
    \item \textbf{Application:} The proposed safety concept is evaluated in an application-oriented simulation environment. The results show that the proposed approach can effectively reduce the risk of accidents and improve the overall safety of automated vehicles.
\end{itemize}
