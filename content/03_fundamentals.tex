% !TeX root = ../root.tex
% -*- root: ../root.tex -*-

\section{Fundamentals}

\subsection{Decision-Making}

Decision-making is crucial in the planning module of a robot, determining commands based on the current situation.
Methods range from graph-based techniques like A*, PRM*, and RRT* to probabilistic and machine learning methods.
This work, however, focuses on rule-based methods, such as \glspl{FSM}, \glspl{BT}, or arbitration graphs, which address decision-making through discrete state or mode transitions.
Refer to \cite{schwartingPlanningDecisionMakingAutonomous2018,yurtseverSurveyAutonomousDriving2020,gammellAsymptoticallyOptimalSamplingBased2021}
for a detailed overview of available approaches.

% Finite State Machines
\textbf{\glspl{FSM}}, first used in hardware design and theoretical computer science,
represent behavior modes with transitions triggered by events \cite{wagnerModelingSoftwareFinite2006}.
Despite their simplicity, \glspl{FSM} scale poorly and are difficult to modify due to the amount of transitions increasing exponentially with the number of states.

% Behavior Trees
\textbf{\glspl{BT}}, initially designed for game development \cite{iovinoProgrammingEffortRequired2022},
have been increasingly used in robotics since 2012 \cite{bagnellIntegratedSystemAutonomous2012,ogrenIncreasingModularityUAV2012}.
They separate behavior decision-making from execution using a tree structure \cite{colledanchiseBehaviorTreesRobotics2018}.
Internal nodes determine selection mechanisms, while leaves describe behaviors and conditions.
Evaluated at a fixed frequency, nodes return their status as running, completed, or failed.
Control flow nodes decide on further evaluations.
%
Condition nodes check if their conditions are met without affecting the environment, while action nodes execute behaviors and return their status.
% By distinguishing between condition and action nodes, the preconditions of a behavior  are decoupled from the actual execution of the behavior.
% To design a safe system, these actions must be linked to reliable condition nodes.

\glspl{BT} generalize many architectures, such as hierarchical \glspl{FSM} and decision trees \cite{colledanchiseHowBehaviorTrees2017} excelling in modularity, hierarchical organization, reusability, responsiveness, and interpretability \cite{colledanchiseBehaviorTreesRobotics2018}.
Their flexibility allows reuse of individual behaviors \cite{bagnellIntegratedSystemAutonomous2012}.
The selection mechanism is intuitive and easy to follow during operation.
However, extensive preconditions can make representations cumbersome, and safety as well as reliability depend significantly on node arrangement.
These drawbacks are addressed by arbitration graphs.

% Arbitration Graphs
\textbf{Arbitration graphs} originated in the context of robot soccer \cite{lauerCognitiveConceptsAutonomous2010},
integrating ideas from Brooks' behavior-based subsumption \cite{brooksRobustLayeredControl1986},
knowledge-based architectures like Belief-Desire-Intention (BDI) \cite{raoAbstractArchitectureRational1992},
and programming paradigms such as object-oriented programming \cite{stefikObjectOrientedProgrammingThemes1985}.

This modular framework is characterized by clear interfaces for transparent decision-making,
using atomic behavior components to represent simple abilities and behaviors.
These modules are combined using arbitrators to create complex system behaviors.

The input to a behavior component is the current situation~$\situation$, provided as sensor data or an interpreted environment model.
If its preconditions are met, the $\invCond$ condition indicates that the behavior component is applicable in the current situation~$\situation$.
In this case, the higher-level instance (i.e., the arbitrator) can instruct the behavior component to compute a command~$\command$.
The currently active behavior component additionally uses the $\comCond$ condition to indicate that its behavior can be continued.
Consequently, the calling instance does not need to know the prerequisites for executing the command returned by the behavior.

Generic arbitrators combine behavior components $\options = \left< o_0, o_1, \dots \right>$,
filter out the applicable subset $\applicableOptions \subseteq \options$ using their invocation and commitment conditions,
and select the best applicable option~$a^*$ for execution.
Arbitration schemes include priority-based, sequence-based, cost-based, and random.
Due to inheritance and a shared interface, arbitrators can include both behavior components and other arbitrators, enabling a hierarchical architecture.

% Comparing Arbitration graphs to BTs
%
%% Return type/Interfaces: command vs. status
%% Preconditions: decoupled from nodes, distributed over tree vs. tightly coupled to nodes
%% Safety: nodes decide on success/failure vs. integrated into selection mechanism with node-independent verifiers

\textbf{Comparing Arbitration graphs to BTs}, both appear similar at first glance, but differ in several key aspects:

% Return Type/Interfaces
\gls{BT} nodes return execution status, while behavior components in arbitration graphs return commands.
%% note: we talk about the AG behavior components, but "arbitration graph behavior components" is a bit lengthy…
%
% While the former adds more flexibility to each nodes actuator-facing interfaces,
While the former adds more flexibility to each node's actuator interfaces,
%
% the latter focuses on a consistent control theory motivated interface $f(\situation) \to \command$.
% the latter focuses on a control theory motivated and verifiable interface $f(\situation) \to \command$.
the latter focuses on a control theory motivated interface ${f(\situation) \to \command}$ % with verifiable~$\command$.
allowing the command to be verified by each arbitrator before executing it in a down-stream module.

% Preconditions
In \glspl{BT}, preconditions are implemented by condition nodes distributed throughout the tree.
%
% In contrast, arbitration graphs couple a behavior component with its preconditions.
In contrast, arbitration graphs require behavior components to define their own preconditions.
%
% This tight coupling ensures that robustness and safety are less dependent on node arrangement.
% This tight coupling is a foundation for robustness and safety being independent of node arrangement.
% This tight coupling leads to robustness and safety being less dependent on the node arrangement.
% Robustness and safety are less dependent on the node arrangement due to this tight coupling.
This tight coupling makes robustness and safety less dependent on the node arrangement.
%
% This tight coupling follows a more object-oriented approach,
% where the behavior component is responsible for its own preconditions.
% adding to overall system robustness and safety.

% Safety
\glspl{BT} rely on each node to decide on its success or failure,
which can lead to safety and reliability issues if not carefully managed.
Arbitration graphs, as enhanced in this paper, integrate safety into the selection mechanism,
using node-independent verifiers.
This functional decomposition reduces the burden on behavior engineers and
allows an easy integration of unsafe, e.g.\ learning-based, behavior components.

% Overall, arbitration graphs offer a more modular and potentially safer approach to decision-making compared to \glspl{BT}, particularly in complex systems where safety and reliability are critical.


\subsection{Fault-tolerant/Robust Systems}

In automated systems, both hardware and software issues can compromise performance and safety.
Causes include programming errors and runtime issues such as optimization problems, making error diagnosis and treatment crucial in system design and during runtime.

Reliable systems aim to maintain performance
despite potential errors, using metrics like error probability, mean lifespan, failure rate, and availability \cite{echtleFehlertoleranzverfahren1990}.
Terminology varies, but disturbances (\enquote{faults}) can lead to errors, potentially causing system failures.

Reliability measures include error prevention, removal, tolerance, and prediction \cite{dubrovaFaultTolerantDesign2013}.
Prevention and removal focus on design and development, while tolerance involves detecting and preventing operational errors.
Prediction estimates future failures.
Error tolerance involves diagnosing and handling errors—restoring faulty components,
computing correct results despite faults, or removing faulty components \cite{echtleFehlertoleranzverfahren1990}.

\subsection{Pac-Man}

\begin{figure}
    \centering
    \includesvg[width=\columnwidth,inkscapelatex=false]{figures/pacman_arbitrator_base}
    \caption{A basic arbitration graph for playing Pac-Man.}
    \label{fig:pacman-arbitrator-base}
\end{figure}

Pac-Man, the 1980 arcade game, involves navigating a maze to eat dots while avoiding ghosts.
Depicted in \cref{fig:entt-pacman} is an open-source implementation\footnote{\url{github.com/indianakernick/EnTT-Pacman}}
of the game that is being used to demonstrate the proposed method.
We split the behavior of Pac-Man into the following behavioral components:
\begin{description}[align=left]
    \item[Avoid Ghosts] Pac-Man tries to increase the distance to the ghosts to avoid being eaten.
    \item[Chase Ghosts] After consuming an energizer, Pac-Man might try to eat the ghosts for extra points.
    \item[Eat Closest Dot] Pac-Man moves towards the dot that is closest to him.
    \item[Change Dot Cluster] Pac-Man moves towards another area with a higher dot density.
\end{description}

The corresponding arbitration graph is shown in \cref{fig:pacman-arbitrator-base}.

