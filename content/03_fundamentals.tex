% !TeX root = ../root.tex
% -*- root: ../root.tex -*-

\section{Fundamentals}

\todo{Fix references}

\subsection{Decision-Making}

Decision-making is crucial in a robot's planning module, determining commands based on the current situation.
Methods range from graph-based techniques like A*, PRM*, and RRT* to probabilistic methods and machine learning for implicit decisions.
This work, however, focuses on rule-based methods, such as \glspl{FSM}, \glspl{BT}, or arbitration graphs, which address decision making through discrete state or mode transitions.

% Finite State Machines
\glspl{FSM}, first used in hardware design and theoretical computer science, represent behavior modes with transitions triggered by events.
Despite their simplicity, \glspl{FSM} scale poorly and are difficult to modify due to the amount of transitions increasing exponentially with the number of states.
% Behavior Trees
\glspl{BT}, initially designed for game development, have been increasingly used in robotics since 2012.
They separate behavior decision-making from execution by leveraging a tree structure.
Internal nodes determine selection mechanisms, while leaves describe behaviors and conditions.
Evaluated at a fixed frequency, nodes return their status as running, completed, or failed.
Control flow nodes decide on further evaluations.

Condition nodes check if their underlying conditions are met without affecting the environment, while action nodes execute behaviors and return their status.
By distinguishing between condition and action nodes, the preconditions of a behavior  are decoupled from the actual execution of the behavior.
To design a safe system, these actions must be linked to reliable condition nodes.

\glspl{BT} generalize many architectures, such as hierarchical \glspl{FSM} and decision trees \cite{colledanchiseHowBehaviorTrees2017} excelling in modularity, hierarchical organization, reusability, responsiveness, and interpretability \cite{colledanchiseBehaviorTreesRobotics2018}.
Their flexibility allows reuse of individual behaviors \cite{bagnellIntegratedSystemAutonomous2012}.
The selection mechanism is intuitive and easy to follow during operation.
However, extensive preconditions can make representations cumbersome, and safety as well as reliability depend significantly on node arrangement.
These drawbacks are addressed by arbitration graphs.

% Arbitration Graphs
The concept of behavior arbitration originated in the context of robot soccer, integrating ideas from Brooks' behavior-based subsumption, knowledge-based architectures like Belief-Desire-Intention (BDI), and programming paradigms such as object-oriented programming.
This approach was thoroughly described in \cite{lauerCognitiveConceptsAutonomous2010} and is summarized in the following.

This modular framework is characterized by clear interfaces for transparent decision-making, using atomic behavior components to represent simple abilities and behaviors.
These modules are combined using arbitrators to create complex system behaviors.

The input to a behavior component is the current situation~$\situation$, provided as sensor data or an interpreted environment model.
If its preconditions are met, the $\invCond$ condition indicates that the behavior component is applicable in the current situation~$\situation$.
In this case, the higher-level instance (i.e., the arbitrator) can instruct the behavior component to compute a command~$\command$.
The currently active behavior component additionally uses the $\comCond$ condition to indicate that its behavior can be continued.
Consequently, the calling instance does not need to know the prerequisites for executing the behavior's command.

Generic arbitrators combine behavior components $\options = \left< o_0, o_1, \dots \right>$,
filter out the applicable subset $\applicableOptions \subset \options$ using their invocation and commitment conditions,
and select the best applicable option~$a^*$ for execution.
Arbitration schemes include priority-based, sequence-based, cost-based, and random.
Due to inheritance and a shared interface, arbitrators can include both behavior components and other arbitrators, enabling a hierarchical architecture.

\subsection{Fault-tolerant/Robust Systems}

In automated systems, both hardware and software issues can compromise performance and safety.
Causes include programming errors and runtime issues such as optimization problems, making error diagnosis and treatment crucial in system design and during runtime.

Research on reliable and fault-tolerant systems aims to design dependable hardware and software despite potential errors, using metrics like error probability, mean lifespan, failure rate, and availability.
Terminology varies, but disturbances (\enquote{faults}) can lead to errors, potentially causing system failures.

Reliability measures include error prevention, removal, tolerance, and prediction.
Prevention and removal focus on design and development, while tolerance involves detecting and preventing operational errors.
Prediction estimates future failures.
Error tolerance involves diagnosing and handling errors—restoring faulty components, computing correct results despite faults, or removing faulty components from the system.

