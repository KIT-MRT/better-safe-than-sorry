% !TeX root = ../root.tex
% -*- root: ../root.tex -*-
\section{Behavior Arbitration for Automated Vehicles}
\subsection{Environment Model}

\begin{itemize}
    \item The environment model describes the current situation for automated driving, including a static representation (planning map, lane graph, route) and a dynamic representation (ego-vehicle state, other traffic participants and obstacles, predictions, perception limits).
    \item The planning map provides information about lane geometries, traffic signs, traffic lights, and drivable areas.
    \item The lane graph describes the relationships between lanes and the applicable traffic rules.
    \item The ego-vehicle state is determined using sensors and filtering techniques.
    \item Other traffic participants and obstacles are detected and tracked using cameras, radar, and/or LIDAR sensors.
    \item The prediction module predicts the future motion of objects.
    \item The worst-case occupancy is determined using reachable set analysis.
    \item The perception limits represent the range of the sensors.
    \item The planned trajectory and the fail-safe trajectory are also stored in the environment model.
\end{itemize}

\subsection{Behavior Blocks}

\begin{itemize}
    \item Abstraction Level: The behavior modules address the tactical level of automated driving, which includes lane changes, parking, and stopping maneuvers.
    \item Maneuver Representation: The behavior modules output a planned trajectory (Soll-Trajektorie) and a fail-safe trajectory, as well as HMI outputs and V2X messages.
    \item Start and Continue Conditions: The start and continue conditions of the behavior modules are derived from the ODDs of the addressed sub-behaviors.
    \item Behavior Generation: The behavior generation includes the planning of the Soll-Trajektorie and the fail-safe trajectory, as well as the determination of HMI outputs and V2X messages.
\end{itemize}

\subsection{Arbitrators}

\begin{itemize}
    \item Arbitrators: The arbitrators are generic and application-independent in their theory and implementation.
    \item Verification Step: A verification step can be integrated into the arbitration to ensure the validity and safety of the actuator output.
    \item Cost Arbitrator: The cost arbitrator is a new type of arbitrator that selects a behavior option based on the expected benefit.
    \item Overview: The paper introduces new arbitrators and provides further fine-tuning of the selection logic.
\end{itemize}

Additional Notes:
\begin{itemize}
    \item The arbitrators are designed to be modular and can be combined to form complex driving maneuvers.
    \item The behavior blocks use a variety of techniques for planning trajectories, including graph search, structured trajectory sets, nonlinear optimization, and reinforcement learning.
    \item The arbitrators are designed to be safe and reliable, and they use a variety of techniques to ensure that they can be realized in real time.
\end{itemize}

\subsubsection*{Cost Arbitrator}

\begin{itemize}
    \item Overview: The cost arbitrator is a new type of arbitrator that can be used to select a behavior option based on the expected benefit.
    \item Start and Continue Conditions: The start and continue conditions of the cost arbitrator are the same as those of the priority arbitrator.
    \item Action: The cost arbitrator filters its options for applicability, determines the action for each applicable option, estimates the associated costs, and sorts these options by their costs. It then selects the cheapest option as the intention and executes it.
\end{itemize}

Additional Notes:

\begin{itemize}
    \item The cost arbitrator can be used to select a behavior option that minimizes the expected travel time, fuel consumption, or other costs.
    \item The cost arbitrator can be used to balance different objectives, such as safety and efficiency.
    \item The cost arbitrator can be used to adapt to changing conditions, such as traffic congestion or weather.
\end{itemize}

