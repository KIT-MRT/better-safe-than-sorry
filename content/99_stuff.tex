% \todo[inline]{Condense and include the following two sections that used to be in fundamentals.}
% \subsubsection*{Behavior Verification}

% Reachable sets:
% \begin{itemize}
%   \item Aims to maintain a provably safe fail-safe trajectory that can be switched to if a collision is imminent
%   \item Fail-safe trajectory is only considered safe if it has no overlap with the worst-case positions of other road users
%   \item Set-point trajectory only needs to ensure that a switch to the fail-safe trajectory is still possible in the next planning interval
%   \item \textbf{Advantages:}
%         \begin{itemize}
%             \item Consistent, intuitive, and efficient
%             \item Fail-safe trajectory concept can be well integrated into a behavior decision-making with fallbacks
%         \end{itemize}
%   \item \textbf{Disadvantages:}
%         \begin{itemize}
%             \item Previous publications make far-reaching assumptions that need to be relaxed for real-world application (e.g., vehicles do not leave their lanes except for lane changes)
%         \end{itemize}
% \end{itemize}

% \subsubsection*{3-layer model for automated driving}
% \begin{itemize}
%     \item \textbf{Strategic layer:} Defines general long-term preferences and goals (e.g., planned route)
%     \item \textbf{Tactical layer:} Generates driving maneuvers (e.g., lane change, stop maneuver) based on current situation and goals
%     \item \textbf{Operational layer:} Actuates the intended maneuver with a high update frequency (e.g., milliseconds)
% \end{itemize}





% \subsection{Environment Model}

% \begin{itemize}
%     \item The environment model describes the current situation for automated driving, including a static representation (planning map, lane graph, route) and a dynamic representation (ego-vehicle state, other traffic participants and obstacles, predictions, perception limits).
%     \item The planning map provides information about lane geometries, traffic signs, traffic lights, and drivable areas.
%     \item The lane graph describes the relationships between lanes and the applicable traffic rules.
%     \item The ego-vehicle state is determined using sensors and filtering techniques.
%     \item Other traffic participants and obstacles are detected and tracked using cameras, radar, and/or LIDAR sensors.
%     \item The prediction module predicts the future motion of objects.
%     \item The worst-case occupancy is determined using reachable set analysis.
%     \item The perception limits represent the range of the sensors.
%     \item The planned trajectory and the fail-safe trajectory are also stored in the environment model.
% \end{itemize}

% \subsection{Behavior Blocks}

% \begin{itemize}
%     \item Abstraction Level: The behavior modules address the tactical level of automated driving, which includes lane changes, parking, and stopping maneuvers.
%     \item Maneuver Representation: The behavior modules output a planned trajectory (Soll-Trajektorie) and a fail-safe trajectory, as well as HMI outputs and V2X messages.
%     \item Start and Continue Conditions: The start and continue conditions of the behavior modules are derived from the ODDs of the addressed sub-behaviors.
%     \item Behavior Generation: The behavior generation includes the planning of the Soll-Trajektorie and the fail-safe trajectory, as well as the determination of HMI outputs and V2X messages.
% \end{itemize}

% \subsection{Arbitrators}

% \begin{itemize}
%     \item Arbitrators: The arbitrators are generic and application-independent in their theory and implementation.
%     \item Verification Step: A verification step can be integrated into the arbitration to ensure the validity and safety of the actuator output.
%     \item Cost Arbitrator: The cost arbitrator is a new type of arbitrator that selects a behavior option based on the expected benefit.
%     \item Overview: The paper introduces new arbitrators and provides further fine-tuning of the selection logic.
% \end{itemize}

% Additional Notes:
% \begin{itemize}
%     \item The arbitrators are designed to be modular and can be combined to form complex driving maneuvers.
%     \item The behavior blocks use a variety of techniques for planning trajectories, including graph search, structured trajectory sets, nonlinear optimization, and reinforcement learning.
%     \item The arbitrators are designed to be safe and reliable, and they use a variety of techniques to ensure that they can be realized in real time.
% \end{itemize}

% \subsubsection*{Cost Arbitrator}

% \begin{itemize}
%     \item The cost arbitrator can be used to select a behavior option that minimizes the expected travel time, fuel consumption, or other costs.
% \end{itemize}
