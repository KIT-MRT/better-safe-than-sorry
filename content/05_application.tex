\section{Safe and Reliable Behavior Arbitration for Automated Vehicles}



%%%%%%%%%%%%%%%%
% Introduction %
%%%%%%%%%%%%%%%%

%% real life applicable
Arbitration graphs can be applied to a wide range of real-world scenarios.
In \cite{orzechowskiDecisionMakingAutomatedVehicles2020a},
they have been used in the context of automated driving.
There, a behavior component represents driving maneuvers of approximately $5$ to $20$ seconds
such as lane changes, parking maneuvers, or crossing intersections.

%% AV → failure states, stuff to consider (which leads to verifiers)
Behavior planning for automated driving comes with a variety of challenges.
The environment is highly complex and dynamic, and safety-critical decisions must be made under real-time constraints.
Among other things, the agent has to consider kinematic and dynamic constraints,
sensor limitations, traffic rules, and vehicle safety.
Therefore, verifying planned behavior and providing fallback options in case of failure are crucial in this domain.

This section summarizes \cite{orzechowskiVerhaltensentscheidungFuerAutomatisierte2023},
which extends \cite{orzechowskiDecisionMakingAutomatedVehicles2020a} by
introducing a safety concept for arbitration graphs in the context of automated driving.



%%%%%%%%%%%%%%%%%%%%%
% Arbitration Graph %
%%%%%%%%%%%%%%%%%%%%%

\subsection{Arbitration Graph --- \textbf{What} to do?}

%% show the graph (simplified, others possible)
\begin{figure}
    \centering
    \includesvg[width=\columnwidth,inkscapelatex=false]{figures/arbitration_graph_simulation_underlined}
    \caption{
        A minimalistic arbitration graph for automated driving as introduced in \cite{orzechowskiDecisionMakingAutomatedVehicles2020a}, extended by fallback layers (underlined).
    }
    \label{fig:arbitration-graph-evaluation}
\end{figure}

\cref{fig:arbitration-graph-evaluation} depicts a simplified arbitration graph for automated driving
that will be used to showcase our safety concept.
It contains three behavior components for basic lateral behaviors \behavior{FollowLane}, \behavior{ChangeLaneLeft} and \behavior{ChangeLaneRight},
a parking behavior \behavior{ParkNearGoal}
and a stopping behavior \behavior{EmergencyStop}.
The arbitration graph for a real-world application would include more behavior components and arbitrators
in order to cover a wide range of driving maneuvers and scenarios.
Refer to \cite{waymo_safety_report_2020} for a detailed discussion about behavior competencies and operational design domains.


\subsection{Behavior Components --- \textbf{How} to do it?}

%%%%%%%%%%%%%%%%%%%%%
% Environment Model %
%%%%%%%%%%%%%%%%%%%%%

\begin{figure}
    \centering
    \includesvg[width=\columnwidth,inkscapelatex=true,pretex=\begin{svgfontonly},apptex=\end{svgfontonly}]{figures/environment_model}

    \caption{%
        Excerpt from the environment model with
        ego vehicle speed,
        last planned intended and fail-safe trajectory
        (blue dashed and red solid lines)
        and sensor range (blue).
        Another vehicle (green)
        with its predicted trajectory (dashed line)
        and its predicted worst-case occupancy (green).
    }
    \label{fig:environment-model}
\end{figure}

%% environment model → short (we use …, but other representations are possible)
In our implementation, each behavior component has access to an environment model (situation~$\situation$) and returns a trajectory (command~$\command$).
%
The environment model contains information
such as the ego vehicle state, other traffic participants, and the planned route.
% The trajectory consists of an intended and a fail-safe part,
% where the fail-safe trajectory is designed to be safe under worst-case assumptions.
\cref{fig:environment-model} illustrates an example scenario with the ego vehicle in blue,
another vehicle in green, and the planned ego trajectory in dashed blue.

%% take one behavior component → explain → short
The $\invCond$ and $\comCond$ conditions of behavior components are derived from the operational design domain of their addressed driving maneuver.
%
%%%%%%%%%%%%%%%%%%%%%%%%
% Example: Lane Change %
%%%%%%%%%%%%%%%%%%%%%%%%
As an example, we will examine the \behavior{ChangeLaneLeft} behavior component.
%% Invocation Condition
Its $\invCond$ condition is true,
if there is a lane to the left of the current lane that the ego vehicle can legally change to and the distance to objects in the adjacent lane is sufficient for a lane change.
%% Committment Condition
The $\comCond$ condition is true, as long as the ego vehicle is actively changing lanes
and the target lane remains clear.
%% Behavior Generation
The trajectory smoothly transitions to the left lane.
% while the fail-safe trajectory brings the ego vehicle back to the original lane.

% Other behavior components for automated driving are described in detail in~[?]\todo{ref Pios Diss}.
See \cite{orzechowskiVerhaltensentscheidungFuerAutomatisierte2023} for other behavior components for automated driving.



% We have to tweak around with the placement of the figure:
%
% Another LATEX 2ε limitation (patched with stfloats or not) is
% that double column floats will not appear on the same page
% where they are defined. So, the user will have to define such
% things prior to the page on which they are to (possibly) appear.
%
% --- IEEEtran_HOWTO.pdf

\begin{figure*}
  \centering
  \includesvg[width=\textwidth,inkscapelatex=true,pretex=\begin{svgfont},apptex=\end{svgfont}]{figures/experiments_crash_combined_topview}
  \caption{
      At position B (see \cref{fig:test-track}),
      the ego vehicle intends to change lanes at $t=\SI{1.9}{\second}$,
      but another vehicle is following close and fast.
      Without verification, the too optimistic \behavior{ChangeLaneLeft} behavior is chosen,
      leading to a collision at $t=\SI{5.3}{\second}$.
      Verification using occupancy predictions prevents this
      (\cref{fig:experiments-safe-occupancies,fig:experiments-safe-arbitration-graph,fig:experiments-safe-timeline}).
  }
  \label{fig:experiments-topview}
\end{figure*}

%%%%%%%%%%%%%%%%%%%%%
%   Verification    %
%%%%%%%%%%%%%%%%%%%%%

\subsection{Verification --- What is considered \textbf{safe}?}

In order to detect unsafe and unreliable behaviors commands,
various aspects need to be considered in the behavior generation process.
%
%% focus on verification (we check …, but other things are possible)
Many aspects could be considered, but here we focus on validity and safety.

The verifier for validity ensures that the planned trajectory fulfills the kinematic and dynamic constraints of the ego vehicle
and adheres to traffic rules.

In order to achieve vehicle safety, we
provide worst-case occupancy predictions of other traffic participants
and extend the behavior command with a fail-safe trajectory \cite{althoffSetBasedPredictionTraffic2016}.
The fail-safe trajectory is designed to be safe under worst-case assumptions
and can be used in fallback layers in case of a failure or too risky situation.
As a result, each behavior component is responsible for generating both
a planned trajectory and a fail-safe trajectory (see \cref{fig:environment-model}).
%% explain overlap fail-safe worst-case occupancies
The safety verifier checks if the occupancy of the fail-safe trajectory does not overlap
with the worst-case occupancies of other traffic participants,
\ie the fail-safe trajectory is collision-free under worst-case assumptions.



% We have to tweak around with the placement of the figure:
%
% Another LATEX 2ε limitation (patched with stfloats or not) is
% that double column floats will not appear on the same page
% where they are defined. So, the user will have to define such
% things prior to the page on which they are to (possibly) appear.
%
% --- IEEEtran_HOWTO.pdf

\begin{figure}[!t]
    \centering
    \includesvg[width=\columnwidth]{figures/experiments_safe_occupancies}
    \caption{
        Predicted worst-case occupancies in green and
        the egos fail-safe trajectory occupancy in blue
        at $t=\SI{1.9}{\second}$.
        An overlap hints towards a risk.
    }
    \label{fig:experiments-safe-occupancies}
\end{figure}

\begin{figure}[!t]
    \centering
    \includesvg[width=\columnwidth]{figures/experiments_safe_arbitration_graph}
    \caption{
        Arbitration graph with verification at $t=\SI{1.9}{\second}$.
        The safety verifier detects that the fail-safe trajectory of \behavior{ChangeLaneLeft}
        overlaps with the worst-case occupancies of the green vehicle
        (\cref{fig:experiments-safe-occupancies}).
        As a result, the \arbitrator{UrbanDriving} arbitrator choses the next best option \behavior{FollowLane}.
    }
    \label{fig:experiments-safe-arbitration-graph}
\end{figure}

%%%%%%%%%%%%%%%%%%%%%
%  Fallback Layers  %
%%%%%%%%%%%%%%%%%%%%%

\subsection{Fallback Layers --- How to always \textbf{drive safe}?}

In order to mitigate unsafe or unreliable behaviors,
we extended the arbitration graph in \cref{fig:arbitration-graph-evaluation}
by the proposed verifiers and multiple fallback layers
highlighted by underlined text.
% This ensures safety and enables graceful degradation.

Since trajectories are planned for a given horizon,
it is feasible to fall back to the last planned trajectory
% in case of intermittent failures.
provided by \behavior{ContinueLastManeuver},
in case of intermittent failures.
% This option is provided as first fallback option by \behavior{ContinueLastManeuver}.
% This is given by \behavior{ContinueLastManeuver}.

Should the last planned trajectory no longer be suitable or safe,
\behavior{FailSafe} can be chosen to execute the fail-safe trajectory described above.

As a last resort, \behavior{EmergencyStop} can bring the vehicle to a full stop.
% This might be necessary if the assumptions of the worst-case occupancy predictions are violated.
This may be necessary if the assumptions of the fail-safe trajectory are violated.
%
Since this is the last resort fallback layer,
it does not go through verification.
Therefore, it is crucial that it is implemented in a simple, deterministic and reliable fashion
without the need for contextual knowledge.



%%%%%%%%%%%%%%%%%%%%%
%    Validation     %
%%%%%%%%%%%%%%%%%%%%%

\subsection{Experiments}

% Context experiments (Simulation Environment / Test Track)
We validate the proposed concept in the automated driving simulator CoInCar-Sim~\cite{naumannCoInCarSimOpenSourceSimulation2018}
using handcrafted but realistic driving scenarios
from our real-world test track in Karlsruhe, Germany.
%% Scenario
We analyze two different use cases
in \cite{orzechowskiVerhaltensentscheidungFuerAutomatisierte2023}:
%
\begin{description}[align=left]
    \item[Ensuring driveability]
        The verifiers and fallback layers lead to stable trajectory commands,
        even under high probability for broken trajectories.
    \item[Guaranteeing vehicle safety]
        The verifiers and fallback layers ensure collision free behavior,
        even if a behavior component provides unsafe commands.
\end{description}

%% what's the challenge, why do we need verification
Here, we focus on the latter, more critical case.
Traditional decision-making approaches select maneuver options solely based on their preconditions.
As a result, they might select unsafe maneuver options,
if the preconditions have been designed too optimistically.
%
\cref{fig:experiments-topview} shows such a scenario, where the ego vehicle wants to change lanes,
but another vehicle follows too closely in the target lane.
We compare the evolution of the scene with and without verification in the decision-making process.


% Results
\subsection{Results}

% We have to tweak around with the placement of the figure:
%
% Another LATEX 2ε limitation (patched with stfloats or not) is
% that double column floats will not appear on the same page
% where they are defined. So, the user will have to define such
% things prior to the page on which they are to (possibly) appear.
%
% --- IEEEtran_HOWTO.pdf

\begin{figure}[!t]
  \centering
  \includesvg[width=\columnwidth,inkscapelatex=false]{figures/experiments_safe_timeline}
  \caption{
      Timeline of the chosen behavior in the case with verification.
      The verifier prevents unsafe behavior commands from being executed.
      The arbitrator falls back to another safe behavior option,
      \behavior{FollowLane} in this case.
  }
  \label{fig:experiments-safe-timeline}
\end{figure}

% Due to a too optimistic $\invCond$ condition,
% an arbitration graph without verification would select \behavior{ChangeLaneLeft}.
To provoke a risky situation,
the $\invCond$ condition of \behavior{ChangeLaneLeft} was made overly optimistic.
As a result, the arbitration graph without verification selects \behavior{ChangeLaneLeft} at $t=\SI{1.9}{\second}$
leading to a collision at $t=\SI{5.3}{\second}$ (\cref{fig:experiments-topview}).

With verification, the safety verifier detects the risk of a collision based on the worst-case occupancies of the other vehicle (\cref{fig:experiments-safe-occupancies}).
Consequently, \behavior{ChangeLaneLeft} fails the verification and
the \arbitrator{UrbanDriving} arbitrator selects \behavior{FollowLane}
as its next best option (\cref{fig:experiments-safe-arbitration-graph}).
This makes the ego vehicle slow down and not change the lane
until the green vehicle has passed (\cref{fig:experiments-safe-timeline}).
A collision has been prevented
while maintaining a smooth driving behavior.
