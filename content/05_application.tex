\section{Application}

\todo[inline]{Condense and include the following two sections that used to be in fundamentals.}
\subsubsection*{Behavior Verification}

Reachable sets:
\begin{itemize}
  \item Aims to maintain a provably safe fail-safe trajectory that can be switched to if a collision is imminent
  \item Fail-safe trajectory is only considered safe if it has no overlap with the worst-case positions of other road users
  \item Set-point trajectory only needs to ensure that a switch to the fail-safe trajectory is still possible in the next planning interval
  \item \textbf{Advantages:}
        \begin{itemize}
            \item Consistent, intuitive, and efficient
            \item Fail-safe trajectory concept can be well integrated into a behavior decision-making with fallbacks
        \end{itemize}
  \item \textbf{Disadvantages:}
        \begin{itemize}
            \item Previous publications make far-reaching assumptions that need to be relaxed for real-world application (e.g., vehicles do not leave their lanes except for lane changes)
        \end{itemize}
\end{itemize}

\subsubsection*{3-layer model for automated driving}
\begin{itemize}
    \item \textbf{Strategic layer:} Defines general long-term preferences and goals (e.g., planned route)
    \item \textbf{Tactical layer:} Generates driving maneuvers (e.g., lane change, stop maneuver) based on current situation and goals
    \item \textbf{Operational layer:} Actuates the intended maneuver with a high update frequency (e.g., milliseconds)
\end{itemize}

